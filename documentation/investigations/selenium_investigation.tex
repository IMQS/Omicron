\documentclass{article}
\title{Investigation of Selenium}
\author{M. Rozenkrantz}
\date{}

\begin{document}
\maketitle

\section{What is Selenium?}
It is an IDE that has a accompanying browser plugin used which records and plays back user interactions with the browser. It was developed due to the increasing number of web-based applications which are being developed, with the purpose of enabling regression testing in a practical manner by automating these tests. This serves to minimize the total time spent manually testing web products. However it is to be noted that it is not advised to use Selenium if there is a time constraint for development as the time taken to write test cases might be greater than the time needed to manually test the product.

\section{Why use Selenium?}
According to András Hatvani (http://www.andrashatvani.com) there are many reasons to use Selenium, these include:
\begin{itemize}
\item{It's desgined for automation of web tests}
\item{It's easy to use}
\item{There are many language APIs to choose from}
\item{There is support for all major browsers}
\item{There are several methods for executing tests}
\item{It can (easily) be incorporated in a Continuous Integration system (CI)}
\item{It is open source}
\end{itemize}


\section{Further reading.}
Please go to \verb#http://docs.seleniumhq.org/# to read more on Selenium.

\end{document}
