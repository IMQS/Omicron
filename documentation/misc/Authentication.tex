\documentclass{article}
\title{Authentication Using Oauth 2 for Twitter and Instagram REST calls}
\author{J\,J\,Martin}
\begin{document}
\maketitle
\section{Introduction}
To connect to most social media sites, they require you to authenticate the user and/or application that wants to connect to the users behalf to show and modify the status and posts on there profile. Before any REST calls can be made to one of these websites you are required to authenticate with the social media server most commonly Oauth is used. 
\section{Twitter Oauth2}
Oauth 2 is used for Twitters Application-only authorization. It is a 3 phase process to allow a application authorization to search through the requested tweets. Once an application is registered on Twitter it will be given a unique \verb+consumer_id+ and \verb+consumer_secret+. These are the two distinct values that represent your application to Twitter which is used to generate access tokens for you application.
\subsection{Phase 1}
The application URL encodes the client-id and client-secret, then concatinates the strings separated by a ``\verb+:+'' which then gets encoded by base 64 encoding known as ``binary to text'' encoding.
\subsection{Phase 2}
The application then makes a ``POST'' request to the authentication server ``\verb+api.twitter.com/oauth2/token+'' using Https (\verb+port 443+) with a header attribute called Authorization and value of \\``\verb+Basic base64_String+'' where \verb+base64_String+ is the binary to text encoded string in phase 1, and one other attributes in the header ``Content-Type'' with a value of \\``\verb+application/x-www-form-urlencoded;charset=UTF-8+''. The body of the request must be  ``\verb+grant_type=client_credentials+''.
\subsection{Phase 3}
Phase 2 should return with a Http status code 200, with an encrypted payload under the gzip encryption (\verb+be sure to decrypt the payload+). Be sure to check that the ``\verb+token_type+'' is ``bearer'' but most importantly you should recieve a ``\verb+access_token+'' this will be used to verify that it is your application that is attempting to access Twitter. Now that an access token has been issued we can make requests to the Twitter API with the key, value pair ``Authorization'' and ``Bearer \verb+access_token+'' in the header of each request.
\section{Instagram user authentication}
Instagram doesn't have a application-only authorisation thus a user based authorisation method is the only way to authenticate with Instagram. This method requires a user to login in and authorise the application to act on behalf of the user for all queries to Instagram. This requires an extra step for authenticating from the Twitter process. Package for the python environment for user authentication can be downloaded at \verb+https://github.com/Instagram/python-instagram+
\subsection{Authentication Process}
The user will click on a login button on the website, which will redirect the user to the instagram website with the following parameters : ``\verb+client_id+'', ``\verb+client_secret+'' and a ``\verb+redirect+''. The ``\verb+client_id+'' and the ``\verb+client_secret+'' are the same as the unique \verb+consumer_id+ and \verb+consumer_secret+ that Twitter uses. Once redirected you need to login to Instagram and authorise the application on the users account, Instagram will then redirect you to the website specified as a parameter (\verb+redirect+) with an ``\verb+code+''.
After receiving the code all three phases of twitter must be done to Instagram to get an access token but with 2 extra parameters specified in the header ``\verb+redirect_url+'' and ``\verb+code+''. The  ``\verb+redirect_url+'' is the call back from Instagram ("the same website specified earlier as \verb+redirect+") and the ``\verb+code+'' is the code acquired from the call back. Once the ``POST'' is made with this header Instagram will respond with a access token for use in the requests to Instagram as a parameter in the query, unlike Twitter which requires it in the header of the request. 


\end{document}
