\documentclass{article}
\author{Shaun Schreiber}
\title{Setting up Apache 2.4 and web.py}
\date{}

\begin{document}
\maketitle

\section{Installation}
\subsection{Apache 2.4}
Type the following commands into the terminal:
\begin{itemize}
\item \verb#sudo apt-get install apache2-dev#
\item \verb#sudo apt-get install apache2-threaded-dev#
\item \verb#sudo apt-get install libapache2-mod-wsgi#
\item \verb#sudo apt-get install libapache2-mod-python#
\end{itemize}
\subsection{web.py}
Type the following commands into the terminal:
\begin{itemize}
\item \verb#sudo apt-get update#
\item \verb#sudo apt-get install python-pip#
\item \verb#sudo pip install web.py#
\end{itemize}

\section{Setting up basic project}
\subsection{Frame work}
Type the following commands into the terminal:
\begin{itemize}
\item \verb#sudo mkdir /var/www/webpy-app/<app name>#
\item \verb#sudo mkdir /var/www/webpy-app/<app name>/logs#
\item Create two files in logs -- access.log and error.log
\item Create a python file in the \verb#/var/www/webpy-app/<app name># directory. Lets call it \verb#main.py#
\item \verb#cd /etc/apache2/sites-available#
\item Create file without an extension with the name same name as your project, lets call it \verb#<project name># . Note the project name could be the same as your \verb#<app name># but it does not have to be.
\end{itemize}
\subsection{Configure Apache 2.4}
Firstly connect the server IP to a name. Note that \framebox{www.mysite.co.za} refers to the server name. Next go to the \verb#/etc# directory and open the file hosts. Add \framebox{127.0.0.1 mysite.co.za} just below localhost. Note that I am using the localhost IP address as my server IP for this demonstration. Save the file -- you will need to have super user privileges to do this.
Navigate to \verb#/etc/apache2/sites-enabled#, open the $<$project name$>$ file and type the following:
\begin{verbatim}
<VirtualHost <server name> or IP:port>
     #ServerAdmin [your admin mail address]
     ServerName superfluous.imqs.co.za
     ServerAlias superfluous.imqs.co.za
     DocumentRoot /var/www/webpy-app/<app name>>
     DirectoryIndex <main python file name>.py
     WSGIScriptAlias /omicron /var/www/webpy-app/<app name>
        /<main python file name>.py/
     ErrorLog /var/www/webpy-app/<app name>/logs/error.log
     CustomLog /var/www/webpy-app/<app name>/logs/access.log combined
     addType text/html .py
<Files <main python file>.py>
        SetHandler wsgi-script
        Options ExecCGI FollowSymlinks
</Files>
<Directory />
        Order Allow,Deny
        Allow From All
        Options -Indexes
</Directory>
<Directory /var/www/webpy-app/<app name>>
        Options +ExecCGI +Indexes +MultiViews +FollowSymLinks
        AllowOverride None
        Order allow,deny
        allow from all
</Directory>
</VirtualHost>
\end{verbatim}
Here is an example.
\begin{verbatim}
<VirtualHost *:80>
     #ServerAdmin [your admin mail address]
     ServerName omicron.imqs.co.za
     ServerAlias omicron.imqs.co.za
     DocumentRoot /home/ubuntu/Omicron/code/src
     DirectoryIndex webservice.py
     WSGIScriptAlias /omicron /home/ubuntu/Omicron/code/src/webservice.py/
     ErrorLog /home/ubuntu/Omicron/code/src/logs/error.log

     CustomLog /home/ubuntu/Omicron/code/src/logs/access.log combined
     addType text/html .py
     Alias /web /home/ubuntu/Omicron/code/src/web
 <Directory />
        Order Allow,Deny
        Allow From All
        Options -Indexes
    </Directory>
<Files webservice.py>
        SetHandler wsgi-script
        Options ExecCGI FollowSymlinks
</Files>
<Directory /home/ubuntu/Omicron/code/src>
        AddHandler server-parsed .js
        Options +ExecCGI +Indexes +MultiViews +FollowSymLinks
        AllowOverride None
        Order allow,deny
        allow from all
</Directory>
</VirtualHost>           
\end{verbatim}
You can test the configuration by typing the following command:
\begin{itemize}
\item \verb#sudo apachectl configtest#
\item \verb#sudo a2ensite <project name>#
\end{itemize}
Only continue if the last line reads "\verb3Syntax OK3".
Navigate to \verb#/var/www/webpy-app/<app name>/#, open \verb#main.py# and type the following:
\begin{verbatim}
import sys
sys.path.append("/var/www/webpy-app/<app name>/")
import web
if app_path: # Apache
    os.chdir(app_path)
else: # CherryPy
    app_path = os.getcwd()
urls = (
    '/(.*)', 'hello'
)
# WARNING
# web.debug = True and autoreload = True
# can mess up your session: I've personally experienced it
web.debug = False # You may wish to place this in a config file
app = web.application(urls, globals(), autoreload=False)
application = app.wsgifunc() # needed for running with apache as wsgi. The reason is because wsgi requires a application method and app.wsgifunc returns the corrent application method for your project.
class hello:
    def GET(self, name):
        if not name:
            name = 'World'
        return 'Hello, ' + name + '!'
if __name__ == "__main__":
    app.run()
\end{verbatim}
Save the file and restart Apache by typing one of the following commands:
\begin{itemize}
\item \verb#sudo service apache2 restart# \textbf{\textsc{or}}
\item \verb#sudo apachectl restart#
\end{itemize}
To test the server; open the browser, and type the following \verb#<server name>/<project name>/hello# into your browser.
If you are looking for more information you can visit the following links.
\begin{itemize}
\item \verb#http://www.hyperink.com/blog/?p=13#
\item \verb#http://webpy.org/cookbook/mod_wsgi-apache-ubuntu#
\item \verb#http://webpy.org/cookbook/cgi-apache#
\item \verb#http://www.youtube.com/watch?v=831OahgMR9k#
\end{itemize}
\end{document}
