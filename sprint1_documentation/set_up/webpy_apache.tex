\documentclass[a4paper]{article}
\author{Shaun Schreiber}
\title{Setting apache 2.4 and web.py}
\begin{document}
\maketitle
\section{Install}
\subsection{Apache 2.4}
Open a terminal and type the following commands.
\begin{itemize}
\item sudo apt-get install apache2-dev
\item sudo apt-get install apache2-threaded-de
\item sudo apt-get install libapache2-mod-wsgi
\item sudo apt-get install libapache2-mod-python
\end{itemize}
\subsection{web.py}
Open a terminal and type the following commands.
\begin{itemize}
\item sudo apt-get update
\item sudo apt-get install python-pip
\item sudo pip install web.py
\end{itemize}
\section{Setting up basic project with Apache and web.py}
\subsection{Setting up the frame work}
Open a terminal and type the following commands.
\begin{itemize}
\item sudo mkdir \verb#/var/www/webpy-app/<app name>#
\item sudo mkdir \verb#/var/www/webpy-app/<app name>/logs#
\item Create two files in logs access.log and error.log
\item Create a python file in the \verb#/var/www/webpy-app/<app name># directory. Lets call it main.py
\item cd \verb#/etc/apache2/sites-available#
\item Create file with no extension with the name of your project, lets call it \verb#<project name># . Note the project name can be the same as your \verb#<app name># but it doesn't have to be.
\item sudo cp \verb#./<project name> ../sites-enabled#
\end{itemize}
\subsection{Configure apache 2.4}
First we need to and our connect our server ip to a name. Note \framebox{www.mysite.co.za} is meant by the server name. So go to the following directory \verb#/etc# and open the file hosts. Now add the follow line just below localhost \framebox{127.0.0.1 mysite.co.za}. Note I am using the localhost IP as my server IP as a demostration.
Save the file Note you will need to be a super user to save.
Navigate to the following directory \verb#/etc/apache2/sites-enabled# and open the <project name> file and type the following.
\begin{verbatim}
<VirtualHost <server name> or IP:port>
     #ServerAdmin [your admin mail address]
     ServerName superfluous.imqs.co.za
     ServerAlias superfluous.imqs.co.za
     DocumentRoot /var/www/webpy-app/<app name>>
     DirectoryIndex <main python file name>.py
     WSGIScriptAlias /omicron /var/www/webpy-app/<app name>/<main python file name>.py/
     ErrorLog /var/www/webpy-app/<app name>/logs/error.log
     CustomLog /var/www/webpy-app/<app name>/logs/access.log combined
     addType text/html .py
<Files <main python file>.py>
        SetHandler wsgi-script
        Options ExecCGI FollowSymlinks
</Files>
<Directory /var/www/webpy-app/<app name>>
        Options +ExecCGI +Indexes +MultiViews +FollowSymLinks
        AllowOverride None
        Order allow,deny
        allow from all
</Directory>
</VirtualHost>
\end{verbatim}
You can test the configuration by typing the following command.
\begin{itemize}
\item sudo apachectl configtest
\end{itemize}
Only continue if the last line reads Syntax OK.
Now navigate to \verb#/var/www/webpy-app/<app name>/# and open your main python file and type the following.
\begin{verbatim}
import sys
sys.path.append("/var/www/webpy-app/<app name>/")
import web
if app_path: # Apache
    os.chdir(app_path)
else: # CherryPy
    app_path = os.getcwd()
urls = (
    '/(.*)', 'hello'
)
# WARNING
# web.debug = True and autoreload = True
# can mess up your session: I've personally experienced it
web.debug = False # You may wish to place this in a config file
app = web.application(urls, globals(), autoreload=False)
application = app.wsgifunc() # needed for running with apache as wsgi. The reason is because wsgi requires a application method and app.wsgifunc returns the corrent application method for your project.
class hello:
    def GET(self, name):
        if not name:
            name = 'World'
        return 'Hello, ' + name + '!'
if __name__ == "__main__":
    app.run()
\end{verbatim}
Save the file and restart apache by typing one the following commands.
\begin{itemize}
\item sudo service apache2 restart
\item sudo apachectl restart
\end{itemize}
To test your server open the browser, modify and type the following \verb#<server name>/<project name>/hello# into your browser.
If you are looking for more information you can visit the following links.
\begin{itemize}
\item \verb#http://www.hyperink.com/blog/?p=13#
\item \verb#http://webpy.org/cookbook/mod_wsgi-apache-ubuntu#
\item \verb#http://webpy.org/cookbook/cgi-apache#
\item \verb#http://www.youtube.com/watch?v=831OahgMR9k#
\end{itemize}
\end{document}
