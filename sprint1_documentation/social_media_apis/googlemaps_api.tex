\documentclass{article}
\title{Google Maps API}
\date{13 June 2013}
\author{Michelle Arzul}

\begin{document}

\maketitle

\section{Introduction}

This document aims to describe the various APIs useful to our purposes that are provided by Google Maps and related products.

\section{Using Google Maps in a web page}

The most straightforward use for Google Maps API is to embed a map into a web page using JavaScript. This essentially sits a script inside a canvas \texttt{div} element. The explicitly defined size of this will determine the size of the map. A number of viewing options (center, zoom, view type, etc) can be specified in a map options variable. By creating a \texttt{Map} object with the canvas element and options as parameters, the map can now be loaded asynchronously by calling the \texttt{initialize} method at an \texttt{onload} event.

For more details:

\noindent\texttt{https://developers.google.com/maps/documentation/javascript/\\tutorial}

\section{Importing data with GeoJSON}

A Maps API application as described above is able to accept an parse new data from various sources:
\begin{itemize}
\item A local file, via XMLHttpRequest
\item A CORS-enabled server
\item JSONP requests
\end{itemize}

GeoJSON is a standard for geographic data. It is a subset of JSON, so anything that can handle JSON can also handle GeoJSON. \footnote{GeoJSON specification: \texttt{http://www.geojson.org/geojson-spec.html}}

Requesting data from local and remote servers is fairly straightforward, provided certain prerequisites are met (for local, must be on the same domain; for remote, must be CORS-enabled).

JSONP makes use of a request script and a callback script which is defined by the target. However this is a risky process, since whatever is returned is used as a script, which can be dangerous.

The data is parsed according to the GeoJSON specification.

There are also a number of third-party utilities that can convert between data standards, such as \texttt{geojason.info}.

For more details:

\noindent\verb#https://developers.google.com/maps/tutorials/data/importing_data#

\section{Using Google Geocoding API}

This API is particularly useful for the purposes being investigated. A geocoding request can return either in JSON or XML format.

The search parameters must include either an address, a lat/lon value, or a component filter (can be an optional parameter if address is specified), plus whether or not the request comes from a device with a location sensor. Optional parameters include bounds, language, region code, and component filters.

For our application and for the sake of uniformity, we are interested in the JSON return format.

One can use this API either for coordinate lookup of an address, or a reverse geocoding (address lookup with coordinates).

This API is very versatile and seems the most useful for our purposes, in conjunction with the data import utilities.

For more details:

\noindent\texttt{https://developers.google.com/maps/documentation/geocoding/}

\end{document}
