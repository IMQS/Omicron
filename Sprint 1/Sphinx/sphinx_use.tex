\documentclass{article}

\title{Using Sphinx}
\author{M. Rozenkrantz}
\date{}

\begin{document}
\maketitle

\section{Sphinx in Python}
The following serves as a summary of the most important things to know about incorporating Sphinx into your Python comments. The full specification for the Python domain can be found at \\
\verb7http://sphinx-doc.org/domains.html#the-python-domain7\,.
\paragraph{How to create function names and parameters}
\begin{verbatim}
.. py:function:: enumerate(sequence[, start=0])

   ...
\end{verbatim}
There are other directives for documenting other types of Python objects, such as \verb7py:class7 and \verb7py:method7.
Please note that \verb7py:7 can be left out as Python is the default domain.

\paragraph{How to add a short description following the function name.}\footnote{The \emph{*} characters which frame 'sequence' serves to make the word sequence appear in \textit{italics}}

\begin{verbatim}
.. function:: enumerate(sequence[, start=0])

   Return an iterator that yields tuples of an index and an item of the
   *sequence*. (And so on.)
\end{verbatim}

\paragraph{Create a link to a specific object within the Sphinx documentation}
\verb7:py:func:`enumerate`7 will create a link to the function enumerate where it appears in the actual Sphinx documentation.

\paragraph{Using Autodoc} Firstly one must make sure that \verb7sphinx.ext.autodoc7 in the \verb7conf.py7 file is assigned to the list \verb7extensions7.

\end{document}