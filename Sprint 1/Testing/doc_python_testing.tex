\documentclass{article}

\title{Python test cases}
\author{}
\date{}

\begin{document}
\maketitle
\section{Python Testing}
For more information and complete examples of how to write unittest test-cases please visit \verb5http://docs.python.org/2/library/unittest.html5$\,$.

\section{Conventions}
\paragraph{Repository conventions} Before committing any changes that have been made, the repository shall first be pulled, and then the commit and push may follow. Commit messages shall be as short as possible while still being as informative as possible. These messages shall not be omitted when committing and pushing to the repository.
\paragraph{Saving of test-cases} Test-cases shall be saved within the repository in the following folder - \verb{\sprint1_code\test{.
\paragraph{Naming of test-cases} Test-cases shall be named as follows - 

\noindent\verb{tests_<module_name_being_tested>.py{$\,$. Note: no capital letters, underscores separated words.


\paragraph{Naming of functions} All functions within a test-case shall be named as follows - 

\noindent\verb{test_<function_your_working_with>_<description_of_what's_being_tested>{

\section{Miscellaneous things that are important}

\paragraph{Help} for manual running of test-cases can be found by running - 

\noindent \verb5 python -m unittest -h5


\end{document}